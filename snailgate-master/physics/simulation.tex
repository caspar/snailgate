\documentclass[12pt]{article}

\renewcommand*\familydefault{\sfdefault}

\usepackage[left=1in,right=1in,top=1in,bottom=1in]{geometry}
\usepackage{graphics,epsfig,graphicx,float,subfigure,color}
\usepackage{algorithm,algorithmic}
\usepackage{amsmath,amssymb,amsbsy,amsfonts,amsthm}
\usepackage{listings}
\usepackage{color} %red, green, blue, yellow, cyan, magenta, black, white

\title{Snailgate: Summary of Simulation}

\begin{document}

\maketitle

\section{Physics Problem}
We are trying to solve the following problem:
\begin{align*}
  m \ddot{u} = R(u) \text{, for each node}
\end{align*}
, where $m$ is the mass, $u(t)$ represents position of the node through time and $R(u)$ is the resulting force on the node. Notice that the force on the node $R(u)$ is affected by the position of all other nodes in the system and also from gravity, water level and others.

Since the mass is constant, we could consider it equal to $1$ and work with the following formula:
\begin{align*}
  \ddot{u} = R(u)
\end{align*}

But this is a second order ODE. To make it easier to solve, we first transform it in a first order ODE by declaring velocity $v$ as $v(t) = \dot{u}(t)$.

Then, we have the following system for each node:
\begin{align*}
  \begin{cases}
    \dot{u} = v\\
    \dot{v} = R(u)\\
  \end{cases} \text{, for each node}
\end{align*}

Now, we have a first order system of ODEs. Notice that this two equations repeat for all nodes in our simulation, so the size of the system depends on the discretization that we use. A fine discretization creates a larger (and probably more difficult) system to solve.

Since we are dealing with a 2 dimensional problem, we need to represent values on $x$ and $y$ directions. A way of doing this is considering $u = (u_x, u_y)$ and $v = (v_x, v_y)$ and using this variables in the system. So, for each node, we end up with four variables.

Consider now that we have a large vector containing all $u$ and $v$ variables for all nodes. We have then:
\begin{align*}
  U = \begin{bmatrix}
    \vdots\\
    u^{(i)}_x\\
    u^{(i)}_y\\
    v^{(i)}_x\\
    v^{(i)}_y\\
    u^{(i+1)}_x\\
    u^{(i+1)}_y\\
    v^{(i+1)}_x\\
    v^{(i+1)}_y\\
    \vdots\\
  \end{bmatrix}
  \text{ and }
  F = \begin{bmatrix}
    \vdots\\
    v^{(i)}_x\\
    v^{(i)}_y\\
    R^{(i)}_x\\
    R^{(i)}_y\\
    v^{(i+1)}_x\\
    v^{(i+1)}_y\\
    R^{(i+1)}_x\\
    R^{(i+1)}_y\\
    \vdots\\
  \end{bmatrix}
\end{align*}

Finally, we can represent our first order ODE system as:
\begin{align*}
U'(t) = F(U)  
\end{align*}

\section{Resolution methods}
With this formulation, it is easy to write Forward and Backward methods for this problems.

\subsection{Forward Euler (Explicit method)}
In the case of Forward Euler, we simply discretize and approximate the derivative on the left side of equation $U'(t) = F(U)$:
\begin{align*}
  \frac{U(t+k) - U(t)}{k} = F(U(t))\\
  \frac{U^{n+1} - U{n}}{k} = F(U^n)\\
  U^{n+1} = U^{n} + k F(U^n)
\end{align*}

Notice that, when solving for time $t+k$ (which corresponds to iteration $n+1$) we can solve $U$ explicitly by simply using the value of $U^n$ and applying $F$ on it.

\subsection{Backward Euler (Implicit method)}
In the case of Backward Euler, we aim to use $F$ applied on the future iteration. Then, it is not possible to solve $U^{n+1} = U(t+k)$ explicitly and we have to solve the system below:
\begin{align*}
  \frac{U(t+k) - U(t)}{k} = F(U(t+k))\\
  \frac{U^{n+1} - U{n}}{k} = F(U^{n+1})\\
  U^{n+1} = U^{n} + k F(U^{n+1})\\
\end{align*}

Notice that if $F$ is linear, meaning that $F(U^{n+1}) = A U^{n+1}$ for some matrix $A$, we would have a linear system $(I - k A)U^{n+1} = U^{n}$.

However, since we are dealing with water pressure and elasticity, which do not give rise to linear forces, we have to solve a nonlinear system:
\begin{align*}
  U^{n+1} - U^{n} - k F(U^{n+1}) = 0\\
  G(U^{n+1}) = 0
\end{align*}

Thus, we are basically searching for values of $U^{n+1}$ for which $G(U^{n+1})$ is $0$. For solving this, we can make use of known literature techniques as the famous Newton method for nonlinear systems.








\end{document}